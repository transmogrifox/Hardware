\documentclass{scrartcl}
\usepackage[utf8]{inputenc}
\usepackage[english]{babel}
\usepackage{amsmath}
\usepackage{amsfonts}
\usepackage{amssymb}
\usepackage{graphicx}
\author{Ryan J Billing}
\title{Design Reference Sheet}
\subtitle{Switched Inductor Power Supplies}
\begin{document}
\maketitle
	\section{System quantities and definitions}
		
		\subsection{Converter types considered}
			The following can be modeled using information contained in this reference sheet.
			\begin{enumerate}
				\item Boost
				\item Buck
				\item Buck-Boost
				\item Flyback (Isolated Buck-Boost)
				\item Forward (Buck)
			\end{enumerate}
		
		\subsection{Control modes}
			The following controls schemes are considered.		
			\begin{enumerate}
			\item Current Programmed Mode (CPM) Control (a.k.a Peak Current) Mode Control
			\item Duty Control (a.k.a Voltage Mode Control)
			\end{enumerate}
		
			Others such as Hysteretic, Constant On Time (COT), Quasi-resonant (QR) can be characterized for steady-state values but dynamic behaviors need to be derived and modified to describe the elements unique to the control scheme.
		\subsection{Operational modes}
		A converter can operate in a mode where energy stored in the inductor during the charging cycle is entirely discharged to the load within each cycle, or it can partially discharge and maintain a certain amount of energy stored between cycles.		
		\begin{enumerate}
			\item \textbf{Continuous Conduction Mode (CCM):} Inductor does not fully discharge
			\item \textbf{Boundary Conduction Mode (BCM):} Inductor fully discharges
			\item \textbf{Discontinuous Conduction Mode (DCM):} Inductor fully discharges
			
			In BCM the inductor finishes discharging immediately at termination of the switching cycle. During DCM operation the inductor discharges and remains idle for a finite period of time between cycles.
		\end{enumerate}	
		\subsection{Definitions for Variables}
		
		From Tan/Middlebrook, "Notation conventions adopted here  are  as  follows.  Capital letters are used  to indicate quantities associated with steady- state, hatted letters are quantities associated with small-signal perturbations, and small letters represent quantities associated with perturbed state,  i.e.,  quantities which are  the sum of capital and  hatted letters."
			
		The following parameters are generally used to express system state variables,
			\begin{align*}
			I_C, i_C, \hat{i}_c &= \text{Peak current control set point for Current Programmed Mode (CPM) control}\\
			I_V, i_V, \hat{i}_v &= \text{Inductor valley current.} \\
			V_{CG},v_{CG},\hat{v}_{cg} &= \text{Voltage across inductor during charging cycle} \\
			V_{DG},v_{DG},\hat{v}_{dg}  &= \text{Voltage across inductor during discharging cycle} \\
			I_{CG}, i_{CG}, \hat{i}_{cg} &= \text{Charging cycle inductor current} \\
			I_{DG}, i_{DG}, \hat{i}_{dg} &= \text{Discharging cycle inductor current}
			\end{align*}
			Items generally used as constants,
			\begin{align*}	
			F_{SW} &= \text{Modulator switching frequency}\\
			T_{SW} &= \frac{1}{F_{SW}}\text{, Modulator switching (sampling) period} \\
			L_m &= \text{Converter magnetizing inductance} \\
			m_c &= \frac{V_{CG}} {L_m}\text{, Slope of current during charge cycle (A/s)} \\
			m_d &= \frac{V_{DG}} {L_m}\text{, Slope of current during discharge cycle (A/s)}\\
			m_{cmp} &= \text{Artificial ramp added to primary current sense signal (slope compensation) (A/s)}\\
			\alpha &= \frac{m_c + m_d} {m_c + m_{cmp}} =  \frac{V_{CG} + V_{DG}} {V_{CG} + m_{cmp} L_m}\\
			D &= \text{Modulator charging cycle duty} \\
			D' &= (1-D) \text{Modulator discharging cycle duty}\\	
			\end{align*}
			
			\subsection{Useful identities}
			\begin{align}
				D &= \frac{m_d}{m_c + m_d} = \frac{V_{DG}}{V_{CG}+V_{DG}} \label{duty_general}\\
				I_{pk} &= I_c\frac{m_c}{m_c+m_{cmp}} \label{control_to_peak_general}
			\end{align}

	\section{Continuous Conduction Mode}
		\subsection{Control-to-Valley-Current Recursive Filter Structures}
		The average currents can be computed as a function of inductor valley current and charge/discharge slopes.  Valley current is easily related to the control input through a linear IIR filter.
		\subsubsection{Peak current control input to valley current}
			\begin{align}
			i_v[n] &=  \alpha i_c[n] + ( 1 - \alpha ) i_v [n-1] - m_dT_{SW}    \label{ivn_cpm}
			\end{align}
		
		\subsubsection{Duty control input to valley current}
			\begin{align}
			i_v[n] &= i_v[n-1] +  D[n](m_c+m_d)T_{SW} - m_dT_{SW}  \label{ivn_dc}	
			\end{align}		
		

	\subsection{Steady State Quantities related to Valley Current}
	This section presents interesting quantities as a function of the valley currents.
	
		\subsubsection{Steady State Valley Current}
			\begin{align}
			I_{v} &=  I_{c} - \frac{T_{SW}m_d} {\alpha} =
			I_c - \bigg(\frac{T_{SW}} {L_m}\bigg) \frac{(V_{CG}+L_m m_{cmp})V_{DG} }{V_{CG}+V_{DG}} \label{ic_to_iv}
			\end{align}
			
		\subsubsection{Steady State Inductor Current Ripple}
			\begin{align}
			\Delta_i &=  \frac{m_c m_d}{m_c + m_d} T_{SW} = \bigg(\frac{T_{SW}} {L_m}\bigg)\frac{V_{CG} V_{DG}}{V_{CG} + V_{DG}}
			\end{align}	
	
		\subsubsection{Steady State Peak Current}
			\begin{align}
			I_{pk} &=  I_v + \Delta_i \label{ripple}
			\end{align}	
		
		\subsubsection{Steady State Cycle Average Currents}
			Average current during the inductor charging cycle.
			\begin{align}
			I_{CG} = \frac{L_m}{2V_{CG}T_{SW}} (I_{pk}^2 - I_v^2) \label{icg_ss}
			\end{align}	
			
			Average current during the inductor discharging cycle.
			\begin{align}
			I_{DG} = \frac{L_m}{2V_{DG}T_{SW}} (I_{pk}^2 - I_v^2) \label{idg_ss}
			\end{align}	
			
			Sanity check (by inspection), input power is equal to output power. Solve (\ref{icg_ss}) for $I_{CG}V_{CG}$ and (\ref{idg_ss}) for $I_{DG}V_{DG}$.  We see both are equal to the same quantity, so the following is also true.
			\begin{align}
			I_{CG}V_{CG} = I_{DG}V_{DG}  
			\end{align}	
	
	\subsection{Steady State Quantities related to cycle average currents}	
	In some cases it is more convenient to calculate converter state variables as a function of input or output current.  For example, if the load current is known, one would want to work backwards through the modulator to determine the valley current, inductor current ripple, and control set points.
	
		\subsubsection{Valley current as function of charge cycle current}
			\begin{align}
			I_{v} = \frac{V_{CG} + V_{DG}}{V_{DG}} I_{CG} - \bigg(\frac{T_{SW}}{2L_m}\bigg)\frac{V_{CG}V_{DG}}{V_{CG} + V_{DG}} 
			\end{align}	
			
			From (\ref{ripple}),
			\begin{align}
			I_{pk} = \frac{V_{CG} + V_{DG}}{V_{DG}} I_{CG} + \bigg(\frac{T_{SW}}{2L_m}\bigg)\frac{V_{CG}V_{DG}}{V_{CG} + V_{DG}} 
			\end{align}	
			From (\ref{ic_to_iv}),

			\begin{align}
			I_{c} = \frac{V_{CG} + V_{DG}}{V_{DG}} I_{CG} - \bigg(\frac{T_{SW}}{2L_m}\bigg)\frac{V_{CG}V_{DG}}{V_{CG} + V_{DG}} + \frac{T_{SW}m_d} {\alpha}
			\end{align}				
				
			\subsubsection{Valley current as function of discharge cycle current}
			\begin{align}
			I_{v} = \frac{V_{CG} + V_{DG}}{V_{CG}} I_{DG} - \bigg(\frac{T_{SW}}{2L_m}\bigg)\frac{V_{CG}V_{DG}}{V_{CG} + V_{DG}} 
			\end{align}				
			From (\ref{ripple}),
			\begin{align}
			I_{pk} = \frac{V_{CG} + V_{DG}}{V_{CG}} I_{DG} + \bigg(\frac{T_{SW}}{2L_m}\bigg)\frac{V_{CG}V_{DG}}{V_{CG} + V_{DG}} 
			\end{align}	
			From (\ref{ic_to_iv}),
			
			\begin{align}
			I_{c} = \frac{V_{CG} + V_{DG}}{V_{CG}} I_{DG} - \bigg(\frac{T_{SW}}{2L_m}\bigg)\frac{V_{CG}V_{DG}}{V_{CG} + V_{DG}} + \frac{T_{SW}m_d} {\alpha}
			\end{align}	

	\section{Boundary Conduction Mode (BCM)}
	Formulas in this section help characterize where the converter transitions from Discontinuous Conduction Mode (DCM) to Continuous Conduction Mode (CCM). The point at which a converter charges and completely discharges the inductor energy concurrent to termination of the switching period is called "Boundary Conduction Mode" (BCM).
		\subsection{Peak current during BCM}
		
			\begin{align}
			I_{pk} = \bigg(\frac{T_{SW}}{L_m}\bigg)\frac{V_{CG}V_{DG}}{V_{CG} + V_{DG}} 
			\end{align}	
			
		\subsection{Steady state average currents during BCM}
			Average inductor currents during charge cycle:
			\begin{align}
			I_{CG} = \bigg(\frac{T_{SW}}{2L_m}\bigg)\frac{V_{DG}^2 V_{CG}}{(V_{CG} + V_{DG})^2} 
			\end{align}			

		   Average inductor currents during discharge cycle:

			\begin{align}
			I_{DG} = \bigg(\frac{T_{SW}}{2L_m}\bigg)\frac{V_{CG}^2 V_{DG}}{(V_{CG} + V_{DG})^2} 
			\end{align}		

	\section{Discontinuous Conduction Mode (DCM)}
	
		\subsection{Steady State Peak Current in DCM}
		
		\begin{align}
		I_{pk} = \sqrt{\frac{2T_{SW}V_{DG}}{L_m}I_{DG}}
		\end{align}	
		
		or referenced to discharge cycle average current,
	
		\begin{align}
		I_{pk} = \sqrt{\frac{2T_{SW}V_{CG}}{L_m}I_{CG}}
		\end{align}	
		
		\subsection{Duty cycle and switch idle time}
		Duty cycle expressed in (\ref{duty_general}), but the quantity $D'$ no longer represents discharge cycle time.  There is a switch conduction interval followed by an interval while the switch is idle (no conduction) during the discharging cycle.
		
		\begin{align}
		D &= \frac{ I_{pk}}{ T_{SW} m_{c}} = \frac{1}{ T_{SW} m_{c}} \sqrt{\frac{2T_{SW}V_{CG}}{L_m}I_{CG}} =  \sqrt{\frac{2 L_m}{T_{SW} V_{CG}}I_{CG}} = \sqrt{\frac{2 L_m V_{DG}}{T_{SW}V_{CG}^2}I_{DG}} \\
		D' &= \frac{ I_{pk}}{ T_{SW} m_{d}} = \frac{1}{ T_{SW} m_{d}} \sqrt{\frac{2T_{SW}V_{DG}}{L_m}I_{DG}}	 = \sqrt{\frac{2L_m} {T_{SW} V_{DG} }I_{DG}} \\
		D_{idle} &= 1 - D - D'
		\end{align}			
				
		\subsection{Steady State Control-to-Average Current in DCM}

		\begin{align}
		I_{CG} = \frac{L_m}{2T_{SW}V_{CG}}I_{pk}^2 \label{dcm_c2cg_ss}
		\end{align}	

		\begin{align}
		I_{DG} = \frac{L_m}{2T_{SW}V_{DG}}I_{pk}^2 \label{dcm_c2dg_ss}
		\end{align}	
	
	\section{RMS Current}
	
	\subsection{CCM}
	Computation for RMS currents during CCM operation, based on the steady state peak and valley current and duty cycle.
	\begin{align}	
		I_{CG.RMS} &= \sqrt{D}\sqrt{I_v^2 + I_v(I_{pk}-I_v) + \frac{1}{3}(I_{pk}-I_v)^2} \label{ccm_rms_cg} \\
		I_{DG.RMS} &= \sqrt{D'}\sqrt{I_v^2 + I_v(I_{pk}-I_v) + \frac{1}{3}(I_{pk}-I_v)^2} \label{ccm_rms_dg}
	\end{align}	
	
	It is possible to show how (\ref{ccm_rms_cg}) and (\ref{ccm_rms_dg}) reduce to the formulas for DCM at BCM, where $\sqrt{D} = \sqrt{\frac{1}{T_{SW}m_c}}$ and $I_v=0$.  The corollary is shown by the same process for discharge cycle currents. 
	
	\subsection{DCM and BCM}
	Valid for both DCM and BCM RMS current calculations. 
	\begin{align}	
		I_{CG.RMS} &= I_{pk} \frac{1}{\sqrt{3T_{SW}m_c}} \\
		I_{DG.RMS} &= I_{pk} \frac{1}{\sqrt{3T_{SW}m_d}} 
	\end{align}
	
	\section{CCM Small signal dynamic behavior}
	For behavior 2 decades below the switching frequency DC values can be used to express the gain of the modulator with a reasonable degree of accuracy.  Often the designer seeks to design a control loop crossover frequency much higher than $F_{SW}/100$ where the modulator dynamic behavior becomes progressively more interesting. 
	
	Behaviors of primary interest:
	\begin{itemize}
		\item $L_m C_{LOAD}$ resonance for duty (voltge-mode) control (see (\ref{ivn_dc}))
		\item $\dfrac{F_{SW}}{2}$ peaking and phase response for CPM control (see (\ref{ivn_cpm}))
		\item Right-Half-Plane-Zero (RHPZ) for boost, buck-boost and flyback topologies 
	\end{itemize}
	
		\subsection{CPM CCM Control to Valley Current Transfer Function}
		The first stage in the system dynamic response is the relationship between peak current control input and the valley current. 
		\begin{equation}
		H_v(z) = \frac {I_v(z)} {I_c(z)} = \frac {\alpha z^{-1}} {1 - (1-\alpha) z^{-1}}  \label{hvz}
		\end{equation}
			
		Evaluating ROC and other stability yields the condition for "academic stability":
		\begin{equation}
		m_{cmp} > \dfrac{m_d - m_c}{2} \label{slope_stab}
		\end{equation}
		
		The term "academic stability" means the system response is guaranteed to settle within a noise-free environment.  Such a designation emphasizes the likelihood that a switched-mode converter system is perpetually perturbed at high frequencies and the observed behavior near academic stability will be indistinguishable from instability.
		
		A useful quantity is the magnitude response at the Nyquist frequency (where peaking occurs):		
		
		\begin{equation}
		\bigg|H_V(z) \bigg|_{z=\frac{\omega_{SW}}{2}} =  \dfrac{\alpha}{2 - \alpha} \label{cpm_peaking}
		\end{equation}
		
		It is usually more useful to pick a desired maximum peaking response (recommended $<6$dB) and then determine what amount of slope compensation is needed to meet the specification.

		\begin{equation}
		m_{cmp} = (m_c + m_d) \frac{1 + |H_v(z)|}{2|H_v(z)|} - m_c \bigg|_{z=\frac{1}{2}\omega_{SW}}
		\end{equation}
		
		Then of course, to express values in dB,
		
		\begin{align}
		G_{dB} &= 20log10\bigg(\frac{\alpha}{2-\alpha}\bigg)\label{Gpeaking}\\
		m_{cmp} &= \frac{1}{2}(m_c + m_d) \frac{1 + 10^\frac{G_{dB}}{20}}{10^\frac{G_{dB}}{20}}  - m_c \label{mcmp_for_gpeaking}
		\end{align}
		
		A reasonable design rule is given below by solving (\ref{mcmp_for_gpeaking}) for $G_{dB} = 6 dB$.
		
		\begin{align}
			m_{cmp} = \frac{1}{4}(3m_d-m_c) \label{mcmp_rule}
		\end{align}
		
		A further interesting observation is if (\ref{mcmp_rule}) is solved for the case $m_c = 3m_d$, the peak current mode controller has 6dB peaking at 25$\%$ duty cycle with no slope compensation. Industry wisdom that slope compensation is needed at 37$\%$ duty corresponds to 12 dB valley current peaking.
		
		\subsection{CCM Valley to Average Current Transfer Function}
		The charging cycle current gain and left-half plane zero (LHPZ).

		\begin{align}
		H_{cg} (s) &= \frac{\hat{I}_{cg}(s)}{\hat{I}_{v}(s)} =\frac{I_{v}}{m_c + m_d}
		\big (\frac{m_d}{I_{v}} + s \big )
		\label{LHPZ_s_raw}
		\end{align}
		
		The discharging cycle current gain and right-half plane zero (RHPZ).
		\begin{align}
		H_{dg} (s) &= \frac{\hat{I}_{dg}(s)}{\hat{I}_{v}(s)} =\frac{I_{v}}{m_c + m_d}
		\big (\frac{m_c}{I_{v}} - s \big )
		\label{RHPZ_s_raw}
		\end{align}

		\subsection{CCM Average to Pulse Current Transfer Function}	
		The following is derived by computing the Laplace Transform of the steady-state current pulse shape.  This forms a low-pass FIR filter with magnitude response rolling off above the Nyquist frequency.  The most interesting information given by these filters is the group delay at higher frequencies.
		
		\subsubsection{Laplace Transform of Charge and Discharge cycle current pulses}	
		
		\begin{align}
		F_{CG}(s) =&   I_{v} \bigg( \frac{1 - e^{-sDT_{SW}}}{s} \bigg)
		+ m_c\bigg( \frac{1 - (1 + sDT_{SW})e^{-sDT_{SW}}}{s^2} \bigg)
		\label{icg_IR_laplace} \\
		\nonumber\\
		F_{DG}(s) =& (I_{v} +m_d D'T_{SW}) \bigg( \frac{1 - e^{-sD'T_{SW}}}{s} \bigg)  e^{-sDT_{SW}}\nonumber\\
		&- m_d \bigg( \frac{1 - (1 + sD'T_{SW})e^{-sD'T_{SW}}}{s^2} \bigg)  e^{-sDT_{SW}}
		\label{idg_IR_laplace}
		\end{align}
		
		\subsubsection{DC gain of pulse transfer function}
		The expressions in (\ref{RHPZ_s_raw}) and (\ref{LHPZ_s_raw}) as well as the control-to-valley current functions give information about the system small-signal gain while (\ref{icg_IR_laplace}) and (\ref{idg_IR_laplace}) include redundancy. Evaluating the limits for $F_{CG}$ and $F_{DG}$ yield the DC gain, the inverse of which can be used to set the overall transfer function DC gain to unity.
		
		\begin{align}
		\lim_{s \to 0} F_{CG}(s) &= \frac{1}{2} D^2 T_{SW}^2m_c + D T_{SW} I_v = \frac{D^2 T_{SW}^2m_c + 2D T_{SW} I_v}{2} \label{lim_FCG} \\
		\lim_{s \to 0} F_{DG}(s) &= \frac{1}{2} D'^2 T_{SW}^2m_d + D' T_{SW} I_v= \frac{D'^2 T_{SW}^2m_d + 2D' T_{SW} I_v}{2} \label{lim_FDG}
		\end{align}
		
		\subsubsection{Normalized pulse transfer function}
		The following will generate the system group delay and high frequency response without affecting the system's small-signal gain that has been captured within the control-to-valley current function and valley-to-average current functions. 
		\begin{align}
		F_{CG.NORM}(s) &= \frac{2}{D^2 T_{SW}^2m_c + 2D T_{SW} I_v} F_{CG}(s)\label{FCG_NORM} \\
		F_{DG.NORM}(s) &= \frac{2}{D'^2 T_{SW}^2m_d + 2D' T_{SW} I_v}F_{DG}(s) \label{FDG_NORM}
		\end{align}
		
\section{DCM small signal dynamic behavior}	
The transfer functions expressed in this section cover DCM converter operation.  In most instances the CCM model is better suited for operation near BCM, however, results will be very similar using either model for characterizing BCM operation.	
	\subsection{CPM DCM control-to-average-current transfer functions }
	Steady state values from (\ref{dcm_c2cg_ss}) and (\ref{dcm_c2dg_ss}) are evaluated for small-signal response by partial derivative, yielding the following small-signal gain quantities.
	
	\begin{align}
	\hat i_{CG} &= \frac{L_mI_{pk}}{T_{SW}V_{CG}}  \hat i_{pk} =  \frac{m_c}{m_c + m_{cmp}}\frac{L_mI_{c}}{T_{SW}V_{CG}} \hat i_{c} \\
	\hat i_{DG} &= \frac{L_mI_{pk}}{T_{SW}V_{DG}} \hat i_{pk} =  \frac{m_c}{m_c + m_{cmp}}\frac{L_mI_{c}}{T_{SW}V_{DG}} \hat i_{c}
	\end{align}
	
	Notice control-to-output gain is expressed in a single step whereas CCM uses control-to-valley current, then valley-to-output current in its transfer function.
	
	During DCM operation there is no memory from cycle-to-cycle so its transfer function is captured as a single FIR system.

	\subsection{Voltage Mode DCM control-to-average-current transfer functions}
	When the modulator input is duty control, the transfer functions are the following.
	
	By using the relationship between duty cycle and peak current, (\ref{icg_ss_dcm}) and (\ref{idg_ss_dcm}) can be used with the following substitution.
	
	\begin{align}
		I_{pk} &= D T_{SW} m_c = D \frac{T_{SW} V_{CG}}{L_m}
	\end{align}
	\begin{align}
	\hat i_{CG} &= \frac{L_m D T_{SW} m_c}{T_{SW}V_{CG}}  \hat d T_{SW} m_c = \frac{ D T_{SW} V_{CG}}{L_m} \hat d \\
	\hat i_{DG} &= \frac{L_m D T_{SW} m_c}{T_{SW}V_{DG}} \hat d T_{SW} m_c = \frac{ D T_{SW} V_{CG}^2}{L_m V_{DG}} \hat d
	\end{align}
	
\subsection{DCM Average to Pulse Current Transfer Function}	
The following is derived by computing the Laplace Transform of the steady-state current pulse shape.  This forms a low-pass FIR filter with magnitude response rolling off above the Nyquist frequency.  The most interesting information given by these filters is the group delay at higher frequencies.

\subsubsection{Laplace Transform of Charge and Discharge cycle current pulses}	

\begin{align}
F_{CG}(s) =&   m_c\bigg( \frac{1 - (1 + sT_{CG})e^{-sT_{CG}}}{s^2} \bigg)
\label{icg_IR_laplace_dcm} \\
\nonumber\\
F_{DG}(s) =&  m_d T_{DG} \bigg( \frac{1 - e^{-sT_{DG}}}{s} \bigg)  e^{-sT_{CG}}\nonumber\\
&- m_d \bigg( \frac{1 - (1 + sT_{DG})e^{-sT_{DG}}}{s^2} \bigg)  e^{-sT_{CG}}
\label{idg_IR_laplace_dcm}
\end{align}
where,
\begin{align}
	T_{CG} &= \frac{I_{pk}} {m_c} \text{ , time to charge inductor during charging cycle}\\
	T_{DG} &= \frac{I_{pk}} {m_d} \text{ , time to discharge inductor during discharging cycle}
\end{align}
	
		\subsubsection{Normalized pulse transfer function}
		The pulse transfer function normalization uses the same formulae as (\ref{FCG_NORM}) and (\ref{FDG_NORM}), setting $I_v=0$.
		\begin{align}
		F_{CG.NORM}(s) &= \frac{2}{T_{CG}^2m_c } F_{CG}(s)\label{FCG_NORM_DCM} \\
		F_{DG.NORM}(s) &= \frac{2}{T_{DG}^2m_d }F_{DG}(s) \label{FCG_NORM_DCM}
		\end{align}
	
\end{document}